

\documentclass[10pt]{article}
%\usepackage{sbc2003} 
%\usepackage[utf8]{inputenc}
%\usepackage[T1]{fontenc}
%\usepackage[brazil]{babel}
%\usepackage{epigraph}
%\usepackage{natbib}
%\setlength{\bibsep}{0.0pt}
%\AtBeginDocument{\RequirePackage{amsfonts}
 % \DeclareSymbolFont{AMSb}{U}{msb}{m}{n}
%	\DeclareSymbolFontAlphabet{\mathbb}{AMSb}}
%\DeclareSymbolFont{rsfscript}{OMS}{rsfs}{m}{n}
%\DeclareSymbolFontAlphabet{\mathrsfs}{rsfscript}
\newcommand{\REM}[1]{}  

\bibliographystyle{plain}

\begin{document}

\hyphenation{}

{\bf Novell Applications of Hardware-Supported Speculative Parallelism}
%
%\author{Guido Araujo\inst{1}, Jos\'e Nelson Amaral\inst{2},}
%
%\address{%\inst{1} %
 %        Instituto de Computac\~ao,
 %        Universidade de Campinas,
 %        Campinas, SP, Brazil\\
 %        \inst{2}%
 %                Department of Computing Science,
 %        University of Alberta,
 %        Edmonton, AB,  Canada
%}

%\maketitle

\begin{description}
\item[Author:] Guido Ara\'ujo
\item[Academic Unit:] Instituto de Computa\,{c}\~ao
\item[Visitor's Institution:] University of Alberta
\item[Country:] Canada
\item[Activities Start Date:] March 13 2017
\item[Number of Days at Unicamp:] 14 days
\item[Schedule of Activities:]  The following activities are planned for this visit:\\
	\begin{itemize}
	\item Research meetings with graduate students and host
	\item Formulation of future joint research strategies.
	\item Co-writing of technical papers
	\item Participation in research seminars
	\item Guest lectures in graduate courses
	\item Presentation of research coloquium
	\end{itemize}

\item[Justification]

This visit will enable the continuation of a very fruitful research collaboration between the host's research group and the visiting researcher's research group. Previous results of this collaboration include the proposal and evaluation of a new mechanism to dynamically schedule transaction execution~\cite{PereiraICPP14}; a new performance evaluation of Intel's version of hardware-supported transactional memory~\cite{PereiraSBAC14,PereiraParCo16}; a very comprehensive study and evaluation of serialization mechanisms for hardware-supported transactional memory~\cite{GaudetSBAC15}; the proposal of a new mechanism to enable the use of hardware support for speculation to enable speculative trace optimization~\cite{SalamancaWAMCA15}; and a proposal to improve Thread-Level Speculation in Hardware Transactional Memory~\cite{SalamancaIPDPS16}. 

This collaboration also has led three Ph.D. students from Unicamp to participate in the sandwich doctoral program with year-long visits to the laboratory of the visiting researcher at the University of Alberta. Also, one of the host's former M.Sc. student was admitted for the doctoral program at the University of Alberta and is currently co-supervised by the host and the visiting researcher. Moreover, this collaboration has led to the creation of a joint Ph.D. and M.Sc. program between  the Instituto de Computa\,{c}\~ao at Unicamp and the Faculty of Science at the University of Alberta.\footnote{Final approval of this program is depending on Unicamp's final approval of the document already approved by the University of Alberta.}

The visiting researcher is participating in an active collaboration with IBM as a consultor for the design of the compiler infrastructure for the newest supercomputers that will be delivered by IBM and NVIDIA to the Department of Energy of the United States of America (USA). These two machines are expected to be the fastest computers in the world when they are delivered. These machines will rely on hybrid hardware --- a combination of CPUs and GPUs --- and on new programming models, faster interconnection networks, and innovative solutions for high-performance computing. At the time of the planned visit, the visiting researcher will be able to discuss several of the novel aspects of the design of this machine, which may influence the delivery of high-performance computing for several years after these machines become available in the market.

The visiting researcher is also an elected director to the Standard Performance Evaluation Corporation (SPEC), the most important organization providing benchmarks to measure computing platforms. He is also a member of the steering committee for the SPEC Research Group. Thanks to these roles the University of Alberta has become a supporting contributor to the effort by the SPEC Open System Group (OSG) to deliver the next version of the SPEC CPU benchmarks. SPEC CPU is the most broadly used and best known benchmark suite to evaluate computer architectures and compilers. In a collaboration with Edson Borin, from Unicamp, the research group of the visiting researcher has provided an extensive evaluation of the benchmark candidates and is in the process of creating alternative inputs to these benchmarks. The proposed visit will also enable the continuation of this collaboration with Edson Borin and significant progress in the writing of articles describing this effort.


\item[Expected Results]

As the publications described above and the record of effective exchange of students evidences, previous visits have been extremely successful and produced important results. Those visits have also contributed for the teaching of computing architecture and compilers at both institutions through the exchange of teaching strategies and material to support teaching. The plan for this visit is to continue this collaboration by both making progress in existing research projects and by defining new research initiatives. 

Besides the successful completion of the activities described in the schedule above, the main goal of the visit is to establish new personal connections with researchers at Unicamp, including graduate students, to define new research directions that can be pursued in the following two years, and to discuss new trends in computer architecture and compiler research and development that may lead to changes in the content taught in these disciplines in the graduate programs at Unicamp and at the University of Alberta.

\end{description}

\section{Details of the activities} \label{activities}

The plan is for a two-week academic visit to the Instituto de Computac\~ao at Unicamp. The following paragraphs describe the planned activities in greater detail.

\subsection{Research colloquium}

The idea is for the visiting professor to give a colloquium to the host department.  Such colloquiums provide a forum for the visiting professor to present his research to a broader group of professors, graduate students, and post-doctoral researchers. Broadly announced research talks allow for researchers that are interested on the topic, but who are not necessarily in direct contact with the host, to participate and follow up with engaging discussions with the visitor. For example, in a previous visit, at such a colloquium given by J. Nelson Amaral at Unicamp, researchers from the Faculty of Engineering and from other universities were in attendance and followed up with interesting discussions about possible future research questions with the visitor.

\subsection{Invited Lectures}

Invited lectures by the visiting scholar to undergraduate and/or graduate classes thought by the hosting scholar. Such lectures allow contact between the students in the host institution and the host scholar, promoting the visit locally, increasing the awareness about the cooperation of the two institutions, and given a more in-depth understanding about the teaching and learning traditions of the host institution to the visiting scholar.

\subsection{Meetings with Groups of Graduate Students}

The ideal format for meetings of the visiting professor with groups of students in the host institution is for a senior graduate student to chair the meeting and for the host professor --- who usually is the supervisor of the students participating --- to not be present. The student chair will give opportunity to each participating student to briefly (in about three minutes) describe his/her research. The visiting professor then follows ups with questions and a discussion of issues or possible lines of inquire. In the past we have found out this meeting format to be extremely effective. It provides the student with the opportunity to learn to summarize the research questions that they are working on and to benefit from the insights and intellectual curiosity of the visiting scholar. It provides the visiting scholar with an overview of the ongoing research in the group. All participating students also benefit from listening to the discussion about the research of their lab mates. This meeting is then followed up with a one-on-one discussion between the visiting scholar and the host scholar so that any issues that may not have been clear --- and often the motivation for each research investigation --- can be clarified between the scholars.

\subsection{One-on-one Meetings}

One-on-one meetings with researchers, professors, and students in the host institution create opportunities to explore research connections between local scholars and the visiting scholars. Such meetings are usually scheduled at the request of one of the local scholars and are an opportunity for consulting with the visiting scholar and for the visiting scholar to have a more in-depth understanding of some of the on-going research at the hosting institution.

\bibliography{local}
\end{document}

%%% Local Variables: 
%%% mode: latex
%%% TeX-master: "tese"
%%% x-symbol-8bits: t 
%%% End:
